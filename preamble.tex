%-----------------------------------------------------------------------------------------------------------------------
%Abkürzungsverzeichnis
\usepackage[printonlyused]{acronym}

%Sprache festlegen auf Deutsch
\usepackage[ngerman]{babel} 

%Quellenverzeichnis
\usepackage[natbib=true, backend=biber, style=authoryear, dashed=false]{biblatex}
\addbibresource{content/bib/bibliography.bib}

%Schriftart
\usepackage{carlito}
\setmainfont{carlito}

%Fortgeschrittene Funktionen für Zitate 
\usepackage{csquotes}

%Positionierung von Tabellen und Abbildungen
\usepackage{float}

%OpenType fonts können geladen werden
\usepackage{fontspec}

%Seitenränder einstellen
\usepackage[left=3.0cm, right=3.0cm, head=2.5cm, bottom=3cm]{geometry}

%Abbildungen
\usepackage{graphicx}  
\graphicspath{{content/images/}}

%Verweise
\usepackage[hidelinks]{hyperref}
\usepackage{cleveref}   %cleveref muss nach hyperref geladen werden

%Header / Footer
\usepackage[headsepline,footsepline,plainfootsepline]{scrlayer-scrpage}
\renewcommand*{\sectionmarkformat}{} %lässt die Nummerierungszahl der Section aus dem Header verschwinden
\clearpairofpagestyles
\ohead{\headmark}
\automark{section}
\ofoot[\pagemark]{\pagemark}
\cfoot[\thesistitel]{\thesistitel}

%Zeilenabstand einstellen
\usepackage[onehalfspacing]{setspace}

%Bilder, Tabellen etc. nebeneinander platzieren
\usepackage{subcaption}

%Für Notizen
\usepackage[colorinlistoftodos,prependcaption]{todonotes}

% Wird gebraucht um Fehlermeldung wegzubekommen
\setlength {\marginparwidth }{2cm} 

%TODOS
\newcommand{\unsure}[2][1=]{\todo[linecolor=red,backgroundcolor=red!25,bordercolor=red,#1]{#2}}
\newcommand{\change}[2][1=]{\todo[linecolor=blue,backgroundcolor=blue!25,bordercolor=blue,#1]{#2}}
\newcommand{\info}[2][1=]{\todo[linecolor=green,backgroundcolor=green!25,bordercolor=green,#1]{#2}}
\newcommand{\improvement}[2][1=]{\todo[linecolor=violet,backgroundcolor=violet!25,bordercolor=violet,#1]{#2}}

%Farben
\usepackage{xcolor}

%-----------------------------------------------------------------------------------------------------------------------
%Counter um die Seiten zu zählen, für römische Zahlen
\newcounter{seitenanzahl}
%Inhaltsverzeichnis ins Inhaltsverzeichnis
\setuptoc{toc}{totoc}
%-----------------------------------------------------------------------------------------------------------------------

