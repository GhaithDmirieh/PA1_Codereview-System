\addsec{Kurzfassung}

Das Codereview bedeutet, dass andere Teammitglieder die Quelltext-Änderungen vom Autor anschauen und kritisieren können. Dieser Prozess ist eine gute Übung sowohl in Open Source als auch in privaten Softwarebereichen. Das heutige Codereview ist weniger formal und leichter als das Codereview, welche in den 70er und 80er Jahren durchgeführt und untersucht wurden. Obwohl das Hauptziel des Reviews die Entdeckung von Fehlern war, hat sich gezeigt, dass dieser Prozess zu anderen Aspekten beiträgt. Diese sind z.B.  Wissenstransfer, Teambewusstsein, Verbesserungsvorschläge und die Schaffung alternativer Problemlösungen. Das alles verbessert die Qualität des Quelltextes \cite{bacchelli2013expectations}.
Zum Reviewen verwenden Entwickler diverse Methoden. Diese unterscheiden sich nach bestimmten Kriterien wie beispielsweise der Ort des Reviews oder die Anzahl der Reviewer. Eine dieser Methoden ist die Verwendung vom \ac{CRS}.
Ein Codereview-System ist ein Tool, das für die Ausführung eines flexibelen Reviews benutzt wird. Mit flexibel ist hier gemeint, dass die Entwickler nicht auf einander in einem bestimmten Ort warten müssen oder, dass sie kein Treffen für das Review organisieren müssen. Es ermöglicht die Besetzter des Codes Regeln aufzustellen, wie z.b. , dass bestimmte Entwickler an bestimmten Teilen des Quelltexts arbeiten oder, dass bestimmte Personen für das Review verantwortlich sind. Das Problem, mit dem sich dieser Arbeit beschäftigt, ist das Fehlen von so einem \ac{CRS} in der Abteilung, da das Review in Präsenzform stattfindet. Das System wird nach bestimmten Kriterien, die für die Abteilung abgestimmt, ausgewählt.