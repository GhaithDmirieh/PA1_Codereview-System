\addsec{Kurzfassung}

Codereview bedeutet, dass andere Teammitglieder die Quelltext-Änderungen vom Autor kritisieren. Dieser Prozess ist eine gute Übung sowohl in Open Source als auch in privaten Softwarebereichen. Der heutige Codereview ist weniger formal und leichter als die Codereview, die in den 70er und 80er Jahren durchgeführt und untersucht wurden. Die Studien zeigten, dass zwar die Fehlererkennung das Hauptziel ist, jedoch traten weniger Fehler als erwartet auf. Stattdessen gab es zusätzliche Vorteile wie Wissenstransfer, Teambewusstsein und die Schaffung alternativer Problemlösungen. Es Wurde festgestellt, dass das Verständnis von Code und Änderungen der Schlüsselaspekt vom Codereview \cite{bacchelli2013expectations}.

\unsure{Gehört die nächsten Zeilen zur Kurzfassung ?}
Die Überprüfung des Quelltext wird das Team Zeit kosten, was aber keinen Nachteil zählt, denn das Team wird keine Zeit sparen, wenn eine Weile später einen Fehler beim Kunden auftritt, was im Nachhinein das Team mehr Zeit als die genommene Zeit für das Review kosten wird, Was mit anderen Worten sagen lässt: Es lohnt sich die Änderungen am Quelltext überprüfen zu lassen \cite{Bjoern}.

\unsure{Die ersten Zeilen sind abstrakt. Mehr bieten meine Quellen nicht an\\Es ist schwierig die Kurzfassung ohne Quellen zu erweitern, denn alles was man sagt, soll auch irgendwo bewiesen sein (Ist auch richtig so) :( }