\addsec{Kurzfassung}

Codereview bedeutet, dass andere Teammitglieder die Quelltext-Änderungen vom Autor kritisieren. Dieser Prozess ist eine gute Übung sowohl in Open Source als auch in privaten Softwarebereichen. Der heutige Codereview ist weniger formal und leichter als die Codereview, die in den 70er und 80er Jahren durchgeführt und untersucht wurden. Die Studien zeigten, dass zwar die Fehlererkennung das Hauptziel ist, jedoch traten weniger Fehler als erwartet auf. Stattdessen gab es zusätzliche Vorteile wie Wissenstransfer, Teambewusstsein und die Schaffung alternativer Problemlösungen. Es Wurde festgestellt, dass das Verständnis von Code und Änderungen der Schlüsselaspekt vom Codereview \cite{bacchelli2013expectations}.