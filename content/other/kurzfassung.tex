\addsec{Kurzfassung}

\acp{CRS} sind vom Alltag aller Programmierer nicht mehr wegzudenken. Ein regelmäßiges verwenden vom \ac{CRS} ist offenbar bei der Qualität des Codes zu sehen, denn es wird bei jeder Änderung des Quellcodes diskutiert und verschieden Meinungen vorgestellt und damit hoffentlich verbessert.

Zum Thema Code review gehört auch andere Methoden ohne \ac {CRS} wie \ac{z. B.}:

\begin{itemize}
\item Ein Treffen für das ganze Team wird regelmäßig organisiert. Je nach bestimmte 	  Bedingungen.
\item Am Ende des Tages werden die Änderungen diskutiert. Das Team muss nicht 			  unbedingt dabei sein, sondern nur die für die Änderungen zuständig sind.
\end{itemize} 

Nachteile dieses Verfahren:
\begin{itemize}
\item Die Änderungen werden zuerst im Quellcode geschrieben,was Fehler enthalten 		  kann, und wahrscheinlich versioniert und danach kommt die Review.
\item Dieser Prozess ist nicht flexibel, denn es fordert die Anwesenheit der für 		  die Änderungen zuständige Personen zusammen mit dem/den Rezensent/en.
\end{itemize}

Andere Methoden werden noch erwähnt und ein wenig erklärt. Allerdings ist das nicht in Details, denn hier geht es hauptsächlich um die \acp{CRS}.
Es wird zuerst das Thema \ac{CRS} dargestellt und werden allgemeine Aspekte vom Thema Code review behandelt.
Es werden danach ein paar \acp{CRS} zur Sprache gebracht und deren Vor- und Nachteile sowie die Eigenschaften jeweils erläutert.
Schließlich wird eins dieser \acp{CRS} vom Autor angeboten und nach Zustimmung der Softwareabteilung wird dieses ausgewählt und in dem unternehmen eingesetzt.
