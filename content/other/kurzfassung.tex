\addsec{Kurzfassung}

Der heutige Codereview ist weniger formal und leichter als die Code-Inspektionen, die in den 70er und 80er Jahren durchgeführt und untersucht wurden. Die Studien zeigten, dass zwar die Fehlererkennung das Hauptziel ist, jedoch traten weniger Fehler als erwartet auf. Stattdessen gab es zusätzliche Vorteile wie Wissenstransfer, Teambewusstsein und die Schaffung alternativer Problemlösungen. Es Wurde festgestellt, dass das Verständnis von Code und Änderungen der Schlüsselaspekt vom Codereview \cite{bacchelli2013expectations}.

Zum Thema Codereview gehören auch andere Methoden wie z. B. :

\begin{itemize}
	\item Reviewsitzung: Das Team trefft sich regelmäßig, je nach Absprache. Diverse Meinungen werden dargestellt.
	\item Stand Up Meeting: Am Ende des Tages werden die Änderungen diskutiert. Das Team muss nicht unbedingt dabei sein, sondern nur die für die Änderungen zuständig sind.
	\item Planung: Entwickler werden zur Review eingeladen, rollen werden verteilt und die Bedingungen werden festgelegt.
\end{itemize} 

Nachteile dieser Methoden:
\begin{itemize}
	\item Die Änderungen werden zuerst im Quellcode geschrieben, was Fehler enthalten kann und danach kommt die Review.
	\item Dieser Prozess ist nicht flexibel, denn es fordert die Anwesenheit der für die Änderungen zuständige Personen zusammen mit den Rezensenten.
\end{itemize}

Andere Methoden werden noch im Abschnitt \ref{subsec:Vorgehensmethoden} erwähnt und ein wenig erklärt. Allerdings nicht in Details, denn hier geht es hauptsächlich um die \acp{CRS}.
