\newpage

\section{Einleitung}
\label{sec:Einleitung}

Dieser Arbeit wurde zum Ziel, ein \ac{CRS}, das gut für die Abteilung passt, aufzubauen, erstellt. Daher müssen ein paar Systeme Vorgestellt und miteinander nach bestimmten Kriterien verglichen werden. Außerdem sollen die ausgesuchte Systeme in der Abteilung getestet und ausgewertet werden.
Es wurden Sechs \acp{CRS} zur Sprache gebracht, deren Vor- und Nachteile sowie deren Eigenschaften wurden jeweils erläutert, davon kamen Bitbucket und Gerrit in die finale Auswahl und wurden getestet \cref{sec:testphase}, gewonnen hat Gerrit weil es die Anforderungen der Abteilung am besten erfüllt und wurde in der Abteilung eingesetzt. Die Begründung dafür, warum wurde sich für das \ac{CRS} Gerrit entschieden ist im \cref{sec:testphase} zu finden.

Es wird in dieser Arbeit zuerst das Thema \emph{Codereview} dargestellt, die Grundlagen, die man zum Verständnis der Arbeit braucht, sind im \cref{sec:Grundlagen} erläutert, danach werden Gründe erwähnt, warum Programmierer der Code nochmal zu besprechen benötigen, was im \cref{subsec:Gründe} zu finden ist. Außerdem werden Vorgehensmethoden erwähnt und allgemeine Aspekte zum Thema Codereview erklärt. Wie hilft das Review bei der Verbesserung der Codequalität ist im \cref{sec:Einfluss des Reviews} erläutert. Im \cref{sec:reviewZeit} wird die Frage beantwortet, ob es sich lohnt, der Code von jemandem anderen zu überprüfen, was auch denjenigen viel Zeit kostet.
Im \cref{sec:Coderview-Tools} werden verschiedene Review-Tools vorgestellt, deren Vor- und Nachteile beschrieben. Dann werden die Eigenschaften sowie die Unterschiede der \acp{CRS} allgemein betrachtet und dann wird argumentiert, warum das \ac{CRS} die sinnvollste Methode zum Reviewen ist. Was für das Unternehmen ein gutes \ac{CRS} macht, wird im Abschnitt \ref{sec:kriterien} erklärt.