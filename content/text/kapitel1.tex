\newpage
\listoftodos

\section{Einleitung}
\label{sec:Einleitung}

In dieser Arbeit wird zuerst das Thema \emph{Codereview} dargestellt, dessen Grundlagen werden im \cref{sec:Grundlagen} erläutert, danach werden Gründe erwähnt, warum Programmierer das Code nochmal zu besprechen benötigen, was im \cref{subsec:Gründe} zu finden ist. Darüber hinaus werden im \cref{subsec:Vorgehensmethoden} Vorgehensmethoden erwähnt und allgemeine Aspekte zum Thema Codereview erklärt. wie hilft der Review bei der Verbesserung der Codequalität ist im \cref{sec:rEinfluss} erläutert. Außerdem wird im \cref{sec:reviewZeit} die Frage beantwortet, ob es sich lohnt, das Code von jemandem anderen zu überprüfen, was auch denjenigen viel Zeit kostet.
Im \cref{sec:Coderview-Tools} werden verschiedene Review-Tools vorgestellt, deren Vor- und Nachteile beschrieben. Danach wird im \cref{subsec:CRS} die Eigenschaften sowie die Unterschiede der \acp{CRS} allgemein betrachtet und dann wird argumentiert, warum das \ac{CRS} die sinnvollste Methode zum Reviewen ist. Was für das Unternehmen ein gutes \ac{CRS} macht, ist im Abschnitt \ref{sec:kriterien} erklärt.

Es werden danach ein paar \acp{CRS} zur Sprache gebracht und deren Vor- und Nachteile sowie deren Eigenschaften jeweils erläutert. Schließlich wurden zwei dieser \acp{CRS} ausgesucht \cref{sec:Ausgesuchte Systeme}, getestet \cref{subsec:testphase} und am Ende ausgewertet \cref{subsec:Auswertung} und nach Zustimmung der Softwareabteilung wird eins davon ausgewählt und in dem unternehmen eingesetzt.\change{Ergebnis erwähnen}
