\newpage

\section{Einleitung}
\label{sec:Einleitung}

Ziel dieser Arbeit: Ein \ac{CRS}, das gut für die Abteilung passt, aufbauen. Daher müssen ein paar Systeme Vorgestellt und miteinander nach bestimmten Kriterien verglichen werden. Außerdem sollen die ausgesuchte Systeme in der Abteilung getestet und ausgewertet werden.

In dieser Arbeit wurde zuerst das Thema \emph{Codereview} dargestellt, die Grundlagen, die man zum Verständnis der Arbeit braucht, sind im \cref{sec:Grundlagen} erläutert, danach wurden Gründe erwähnt, warum Programmierer das Code nochmal zu besprechen benötigen, was im \cref{subsec:Gründe} zu finden ist. Außerdem wurden im \cref{subsec:Vorgehensmethoden} Vorgehensmethoden erwähnt und allgemeine Aspekte zum Thema Codereview erklärt. wie hilft das Review bei der Verbesserung der Codequalität ist im \cref{sec:Einfluss des Reviews} erläutert. Im \cref{sec:reviewZeit} wurde die Frage beantwortet, ob es sich lohnt, das Code von jemandem anderen zu überprüfen, was auch denjenigen viel Zeit kostet.
Im \cref{sec:Coderview-Tools} wurden verschiedene Review-Tools vorgestellt, deren Vor- und Nachteile beschrieben. Dann wurde im \cref{subsec:CRS} die Eigenschaften sowie die Unterschiede der \acp{CRS} allgemein betrachtet und dann wurde argumentiert, warum das \ac{CRS} die sinnvollste Methode zum Reviewen ist. Was für das Unternehmen ein gutes \ac{CRS} macht, ist im Abschnitt \ref{sec:kriterien} erklärt. Es wurde danach ein paar \acp{CRS} zur Sprache gebracht und deren Vor- und Nachteile sowie deren Eigenschaften jeweils erläutert. Schließlich wurden zwei dieser \acp{CRS} ausgesucht \cref{sec:Ausgesuchte Systeme}, getestet \cref{sec:testphase} und nach Zustimmung der Softwareabteilung wurde das \ac{CRS} Gerrit \ref{subsec:Gerrit} ausgewählt und im unternehmen eingesetzt. Die Begründung dafür, warum wurde sich für das \ac{CRS} Gerrit entschieden ist im \cref{subsec:Auswertung} zu finden.