\section{Codereview-Tools}
\label{sec:Coderview}

\unsure{IDE Tools und  Static analysis tools erwähnen}
\unsure{Beschreibe die Arten von CRSs 1) post-commit 2) pre-commit 3) pull-requests}
\blindtext
\blindtext

\section{Codereview-Systeme, die Git unterstützen}
\label{sec:CRS-Git}

\subsection{Bitbucket}
\label{subsec:Bitbucket}

Bitbucket ist ein Quellcode-Management-System der das \ac{VCS} Git unterstützt. Bitbucket unterstützte nach seinem Entwurf als \ac{VCS} nur Mercurial. Erst in 3 Jahren später wurde um die Unterstützung von Git erweitert. Am 1. Juni 2020 wurde die Unterstützung von Mercurial vollständig eingestellt.

\begin{itemize}
	\item \textbf{Art der Review}: \textit{pre- and post-commit} Und die erfolgt durch eine Pull-Anfrage.
	\item \textbf{Entwickler}: Bitbucket wurde 2007 vom Jesper Nøhr entwickelt und 2010 von Atlassian gekauft. 
	\item \textbf{Reviews Workflow}: Eine Pull-Request kann in 3 verschiedenen Weisen gemacht werden:
		\begin{itemize}
			\item Feature-Branch-Workflow.
			\item Git-flow-Workflow.
			\item Forking-Workflow.
		\end{itemize}
\end{itemize}

Die Abbildung \ref{fig:Forking-workflow} zeigt die Forking-Workflow mit Pull-Anfrage.

\begin{figure}[H]
	\centering
	\includegraphics[width=1.0\textwidth]{Bitbucket_Forking-Workflorw}
	\caption[Bitbuckets Ablauf]{Pull-Request.\\ Quelle: eigene Darstellung}
	\label{fig:Forking-workflow}
\end{figure}		  

\begin{description}
	\item [Vorteile:] \hfill
	\begin{enumerate}
		\item Unbegrenzte Anzahl von private Repositories.
		\item Hosting in Cloud oder auf eigenem Server.
		\item \ac{CI}/\ac{CD} sind in Bitbucket integriert.
		\item Erstellen von einer Merge-Checkliste mit zugeordneten Genehmigern.
		\item Für kleine Teams (Bis 5 Personen) bestehen keine Kosten.
		\item Nettes Design.
	\end{enumerate}
	
	\item [Nachteile:] \hfill
	\begin{enumerate}
		\item Es besteht keine Möglichkeit, um bestehende gemergte commits zu reviewen.
		\item Review ist nur vor dem Zusammenführen von einem Feature-Branch mit dem Master-Branch Möglich.
		\item Um zu hosten in Bitbucketsever steht für mehr als 5 Personen einen monatlichen Beitrag und für Self-hosten (auf eigenem Server) besteht eine einmalige Zahlung. Der 
			Preis für beide Varianten variiert je nach Personenanzahl.
	\end{enumerate}
\end{description}

\subsection{Crucible}
\label{subsec:Crucible}

Crucible ist eine webbasierte Anwendung, die eine Unkomplizierte, formelle Codereview anbietet. Sie unterstützt die \acp{VCS} SVN, Git, Mercurial, CVN, perforce.

\begin{itemize}
	\item \textbf{Art der Review}: \textit{pre- and post-commit}
	\item \textbf{Entwickler}: Crucible ist ein Atlassian Produkt.
	\item \textbf{Workflow}: Es gibt verschiedene Weisen, wie man eine Review in Crucible durchführen kann. Allerdings werden sich diese nach den an der Review beteiligten Personen
		 unterscheiden, diese sind:
		\begin{itemize}
			\item Der Moderator. Er ist die Person, die die Review startet und für das Schließen der Review verantwortlich ist.
			\item Die Reviewer
			\item Der Autor
		\end{itemize}
		Beispiele:
		
		\begin{enumerate}
			\item One-to-One Reviews: Der Autor startet die Review (Der Autor ist also in diesem Workflow der Moderator), danach kommentiert der Reviewer und der Autor kann auf diese
				Kommentar reagieren. Das kann sich mehrmals wiederholen bis die Review fertig ist und der Autor diese schließt.\\
				Die Abbildung \ref{fig:one-to-one-workflow} beschreibt diesen Workflow.
				\begin{figure}[H]
					\centering
					\includegraphics[width=1.0\textwidth]{one-to-one-review}
					\caption[Crucible: one-to-one-review]{Crucible Ablaufsmöglichkeit,\\ Quelle: \cite{Crucible}}
					\label{fig:one-to-one-workflow}
				\end{figure}
				
			\item Formal group reviews: Der Autor startet die Review und ladet die Reviewer ein. So können die Review diese diskutieren und dementsprechend bekommen die Reviewer 
				die Antworten vom Autor (Der Autor ist in dem Fall auch der Moderator). Jeder Diskussionspunkt wird vom Moderato festgelegt. So bald die Review zu Ende kommt,
				fasst der Moderator die Review zusammen und schließt sie.\\
				Die Abbildung \ref{fig:Formal-group-review} zeigt diesen Ablauf.
				\begin{figure}[H]
					\centering
					\includegraphics[width=1.0\textwidth]{Formal-group-reviews}
					\caption[Crucible: Formal group review]{Crucible Ablaufsmöglichkeit,\\ Quelle:\cite{Crucible}}
					\label{fig:Formal-group-review}
				\end{figure}
		\end{enumerate}
		
\end{itemize}

\begin{description}
	\item [Crucibles Vorteile:] \hfill
	\begin{enumerate}
		\item Review des Inhalts ist flexibel.
		\item Diskussionsrunden um in Bestimmten Quelltextzeilen oder ein bestimmtes Commit sowie einen gesamten Änderungssatz zu Kommentieren.
		\item Nachverfolgen, Maßnahmen ergreifen in Bezug auf das, was der Autor wichtig findet.
		\item Ansicht vom Reviewstatus und wer möglicherweise Überprüfungen aufhält.
		\item Berichte über Stellen im Quelltext, die noch nicht überprüft wurden.
		\item unterstützt nicht nur Git sonder SVN auch.
	\end{enumerate}
	
	\item [Crucibles Nachteile:] \hfill
	\begin{enumerate}
		\item Die Review muss manuell in Git eingefügt werden(Mergen ist nicht möglich).
		\item Schwer zu erzwingen.
	\end{enumerate}
\end{description}

\subsection{Helix TeamHub}
\label{subsec:HelixTeamHub}

Helix TeamHub ist eine webbasierte Plattform, die Hosten und Management von Repositories ermöglicht. Sie unterstützt Mercurial, Git, SVN, Maven, Ivy, Docker und mehr.

\begin{itemize}
	\item \textbf{Art der Review}: \textit{pre- and post-commit}.
	\item \textbf{Entwickler}: Perforce. 
	\item \textbf{Reviews Workflow}: Die Review kann in Helix Teamhub eingerichtet werden, so kann das Mitglied, das die Review erstellte, entscheiden, wer die Reviewer
		 sind und welche Änderungen in welchem Branch überprüft werden sollen.
		 Ein Beispiel für einen möglichen Workflow:
		 
		  Git-Feature-Branch-Workfow: Der Ablauf ähnelt sich dem Feature-Branch Workflow von Bitbucket in Abschnitt \ref{subsec:Bitbucket}.
		  Es wird zuerst ein Projekt erstellt und zu diesem Projekt wird ein Repository hinzugefügt. Durch git wird ein Branch in diesem Repository
		  erstellt. Der Entwickler arbeitet an diesem Branch. Sobald er eine Review braucht, erstellt er eine. In den Einstellungen sucht man den Branch aus, dann werden Reviewer
		  ausgewählt. In der Review sind die Kommentare auf Zeilenebene möglich. So bekommt der Autor Antworten und Verbesserungsvorschläge. Sobald die Review fertig ist,
		  können die Änderungen automatisch in der Master Branche zusammengeführt werden, was sich durch einen Klick erfolgen lässt. 
\end{itemize}

\begin{description}
	\item [Vorteile:] \hfill
	\begin{enumerate}
		\item Workflows sind flexibel, so dass die Administratoren diese einrichten können.
		\item Schnelle Übersicht über alle Branchen eines Repositorys.
		\item Zugriffsverwaltung, so kann sich der Admenstrator entscheiden, wer an einer repository arbeitet und welche Rechte hat.
		\item Side-by-side Änderungsvergleichen.
		\item Der Prozess der Review kann kontrolliert werden. Die Review kann auf bestimmte Personen beschränkt werden.
		\item Mergen von Reviews erfolgt automatisch.
		\item Helix Teamhub unterstützt \ac{CI}/\ac{CD} Tools.
		\item Kostenlos für Teams bis 5 Mitglieder.
	\end{enumerate}
	
	\item [Nachteile:] \hfill
	\begin{enumerate}
	\item Für die kostenlose Variante steht totale Speichergröße von 1GB in Cloud zu Verfügung.
	\item Hosten auf eigenem Server ist nicht möglich.
	\item Die Abspeicherung von Daten auf dem Server variiert je nach dem jährlichen Beitrag.
	\end{enumerate}
\end{description}


\subsection{Gerrit}
\label{subsec:Gerrit}

In diesem Abschnitt handelt es sich um ein \ac{CRS}, das sich von den bis jetzt in den Abschnitte \ref{subsec:Bitbucket},\ref{subsec:Crucible} und \ref{subsec:HelixTeamHub} dargestellten \acp{CRS} stark unterscheidet. Gerrit ist ein \ac{CRS}, das nur mit dem \ac{VCS} Git arbeitet und Benutzers Änderungen nicht direkt ohne Kontrolle in den Server gepusht werden lässt. Außerdem ist Gerrit auch eine Webbasierte Anwendung, in der die Review geschieht.

\begin{itemize}
	\item \textbf{Art der Review}: \textit{pre-commit}
	\item \textbf{Entwickler}: Google.
	\item \textbf{Workflow}: Gerrit hat nur einen Reviews Ablauf, der nach diesen Schritten erfolgt:
		\begin{enumerate}
			\item Projekt für die Repository erstellen. Für die Erstellung vom Gerrit-Projekt gibt es 4 Methoden:
			\begin{itemize}
				\item Direkt im Gerrits Web-Plattform.
				\item via REST endpoint. 
				\item via SSH command.
				\item Manuell durch git.
			\end{itemize}
			Hinweis: Nicht jeder Benutzer kann ein Gerrits Projekt erstellen, denn das braucht die Admins Rechte.		
			
			\item Features entwickeln, dann die commiten und pushen. In diesem Fall pusht man die Änderungen nicht in den Master Branch, sondern in einen Gerrits Branchen.
				Gerrit benutzt Dieser Branch für die Review. Dementsprechend kommt man mit \textit{git diff} (git Befehl, um alte und neue Änderungen zu vergleichen) zum Vergleich
				zwischen den Änderungen im Arbeitsverzeichnis und den Änderungen auf der Gerrits Branche.
			\item Nach pushen wird einen Link erzeugt, den zur Webseite führt. Auf dieser Webseite kann der Autor: 
			\begin{itemize}
				\item Diffs anzeigen lassen.
				\item Reviewer hinzufügen.
				\item Kommentare, Anmerkungen an bestimmten Code-Zeilen für die Reviewer hinterlassen.
			\end{itemize}
			Die Abbildung \ref{fig:Gerrit-Reviews Webseite} soll einen allgemeinen Überblick über die Webseite geben.
			
			\begin{figure}[H]
				\centering
				\includegraphics[width=0.9\textwidth]{Gerrit-Reviews Webseite}
				\caption[Gerrit]{Reviews Webseite,\\ Quelle: \cite{Gerrit}}
				\label{fig:Gerrit-Reviews Webseite}
			\end{figure}
						
			\item Die Reviewer können entweder die Review manuell finden oder bekommen die eine E-Mail(wenn sie vom Autor als Reviewer gemerkt sind), dass eine Review
				 gestartet wurde. Die Reviewer schauen die Änderungen, antworten auf sie und am Ende bewerten sie. Die Bewertung ist in Gerrit vorgeschrieben.
				 Dann passiert Folgendes:
				 \begin{itemize}
				 	\item Sind die Änderungen fehlerhaft oder fehlt etwas, was der Autor beachten musste, so fordern die Reviewer den Autor, die Änderungen zu bearbeitet
				 		und nochmal in den Gerrits Branch zu pushen.
				 	\item Erfüllen die Änderungen die Anforderungen, So ist die Review geschlossen.
				 \end{itemize}
			\item Darüber hinaus fragt Gerrit nach Bestätigung, dass die Änderungen getestet wurden, dass die also funktionieren. Testen kann entweder 
			\begin{itemize}
				\item Manuell. Die Änderungen fetchen und lokal testen
				\item Oder durch Nutzen von \ac{CI} Tools.
			\end{itemize}
			\item So bald die Review fertig ist und die Features getestet wurden, werden die Änderungen einschließlich der Review je nach den Projekts Einstellungen in
				den Master Branch gepusht.
		\end{enumerate}
		
\end{itemize}

\begin{description}
	\item [Gerrits Vorteile:] \hfill
	\begin{enumerate}
		\item Gerrit kann multiple Repositories beaufsichtigen.
		\item Clean history \unsure{Erkläre warum (git push --f)}
		\item Jede Repository kann unendliche Anzahl von Branchs enthalten.
		\item Automatisches Mergen mit der Git-Repository nach Überprüfung der Änderungen ist möglich.
		\item \ac{CI}/\ac{CD} können in den Workflow eingebaut werden.
		\item Benutzerfreundliche Webschnittstelle zum Reviewen.
		\item Kostenlos \& Open-Source-Projekt.
	\end{enumerate}
	
	\item [Gerrits Nachteile:] \hfill
	\begin{enumerate}
		\item Mit Gerrit ist die Review und das Testen der neuen Änderungen Pflicht.
		\item Nur Administratoren können Projekte in Gerrit hinzufügen.
		\item Gerrit arbeitet ausschließlich mit Git.
	\end{enumerate}
\end{description}

\subsection{RhodCode}
\label{subsec:RhodeCode}

RhodCode ist ein Quellcode-Management, das die \acp{VCS} Git, Mercurial unterstützt. Es ist auch webbasiert also bietet auch eine Web-Schnittstelle, in der die Review geschieht.
RhodCode bietet eine freie Edition "RhodeCode CE" (Community Edition) sowie zwei kostenpflichtige Editions "RhodeCode EE" (Enterprise Edition) und RhodeCode Cloud (Beta).

\begin{itemize}
	\item \textbf{Art der Review}: \textit{pre- and post-commit}
	\item \textbf{Entwickler}: RhodeCode.
	\item \textbf{Workflow}:
	\begin{itemize}
		\item Die Review kann durch eine Pull-Anfrage gestartet werden. Die Abbildung \ref{fig:Forking-workflow} vom Abschnitt \ref{subsec:Bitbucket} zeigt den Ablauf
			eine Pull-Anfrage. Also der Entwickler arbeitet in einem von Master-Branche abgespalteten-Branch macht Änderungen und sendet eine Pull-Anfrage so bald er fertig ist. 
			So Geschieht die Review dementsprechend vor dem Mergen mit dem Master-Branch.
		\item Oder kann der Reviewer sich die Änderungen in Commits-Historie anschauen und bestimmte Code-Zeile kommentieren und den Autor erwähnen, indem er seinen Namen nach einem 
			\textbf{@} schreibt. So bekommt der Autor eine E-Mail, dass jemand seine Änderungen überprüft hat.
	\end{itemize}
\end{itemize}

\subsubsection{RhodeCode CE}
\label{subsubsec:RhodeCode CE}

\begin{description}
	\item [Vorteile] \hfill
		\begin{enumerate}
			\item Ein kostenloses- und Opensource-Projekt.
			\item Unbegrenzte Anzahl von Benutzern.
			\item Benutzerzugriffskontrolle.
			\item Dateisuche in allen Repositories sowie Volltextsuche \& Indizierung vom Quellcode.
			\item Grundlegende Integrationen einschließlich E-Mail, Slack und Hipchat.
		\end{enumerate}
	\item [Nachteile] \hfill
		\begin{enumerate}
			\item Die diffs lassen sich nur inline angezeigt werden.
			\item Einschränkungen der Review sind nicht möglich.
			\item RhodeCode CE unterstützt keine \ac{CI}/\ac{CD} Tools.
			\item Hosting geht nur auf eigenem Server.
		\end{enumerate}
\end{description}

\subsubsection{RhodeCode EE}
\label{subsubsec:RhodeCode EE}

\begin{description}
	\item [Vorteile] \hfill
		\begin{enumerate}
			\item Hat alle Feature der RhodeCode CE im Abschnitt \ref{subsubsec:RhodeCode CE}.
			\item Die diffs lassen sich inline oder side-by-side angezeigt werden. 
			\item Die Review unterstützt Live-Chat für das Quellcode vor Ort sowie Feedback in allen Repositories.
			\item Regeln für Review stellen. z. B. Wer die Änderungen überprüfen kann.
			\item Sperren von Repositories; Admins können Repositories sperren, so dass keine Änderungen mehr von anderen Benutzern in die Master-Branche gepusht sind.
			\item Dateien, Projekte und Repositories ordnen, strukturieren und unbegrenzt \, verschachteln.
			\item Unterstützt die Migration vom \ac{VCS} SVN zu Git.
			\item RhodeCode EE unterstützt \ac{CI}/\ac{CD} Tools.
			\item Volltextsuche nach Umgebungen mit sehr großen Codebasen. Es kann Terabyte von Daten verarbeiten.
		\end{enumerate}
	\item [Nachteile] \hfill
		\begin{enumerate}
			\item Verfügbar nur für mindestens 10 Benutzer mit einem jährlichen Beitrag von 75\$ pro Benutzer.
			\item Kein open-source Projekt.
			\item Hosting geht nur auf eigenem Server.
		\end{enumerate}
\end{description}

\unsure{RhodeCode Cloud (Beta) ?}

\subsection{Github}
\label{Github}

Github ist eine webbasiert Plattform für das Management vom Quelltext. Github arbeitet ausschließlich mit dem \ac{VCS} Git. Es bietet die Gelegenheit, unbegrenzte Anzahl von public/private Repositories (Private Repositories erst seit 8. Januar 2019) auf Githubs Server zu hosten. Außerdem gehört Github seit Juni. 2018 Microsoft.

\begin{itemize}
	\item \textbf{Art der Review}: \textit{pre- and post-commit}
	\item \textbf{Entwickler}: Tom Preston-Werner, Chris Wanstrath, P. J. Hyett.
	\item \textbf{Workflow}: Die Review funktioniert in Github durch eine Pull-Anfrage, wie beim Bitbuckets Workflow in \ref{subsec:Bitbucket} arbeitet der Entwickler an einem
		speziellen Branch, macht Änderungen und startet eine Pull-Anfrage und wählt die Reviewer, die eine Nachricht bekommen, dass sie zur Überprüfung gefragt sind. Nach Annahme der
		Änderungen, können diese automatisch mit dem Master-Branch zusammengeführt werden.
\end{itemize}

Github bietet 4 Pläne. Davon werden nur 2 besprochen:

\subsubsection{Github Free}
\label{subsubsec:Free}

Die Kostenlose Variante für die Community.
\begin{description}
	\item [Vorteile:] \hfill
	\begin{enumerate}
		\item Seit 14.April.2020 kann unbegrenzte Anzahl von Benutzern nicht nur an öffentlichen Repositories zusammenarbeiten sondern auch an privaten Repositories. 
		\item Kostenlose Nutzung von Quellcode-Test-Tools(\ac{CI}/\ac{CD}) für die öffentlichen Repositories und 2,000 Minuten im Monat für private Repositories.
		\item 500 MB Speicher für GitHub-Pakete und unbegrenzte Speicher für öffentliche Repositories.
		\item Codereview-Tools: Änderungsvergleich anzeigen lassen, Pull-Anfragen \dots
		\item Automatisches Zusammenführen nach Genehmigung der Review.
	\end{enumerate}
	\item [Nachteile:] \hfill
	\begin{enumerate}
		\item Einschränkungen auf Branchen sind nicht möglich.
		\item Die Review kann nicht auf bestimmte Benutzer beschränkt werden.
		\item Öffentliche Repositories haben mehr Features als die private Repositories.
	\end{enumerate}
\end{description}

\subsubsection{Github Team}
\label{subsubsec:Team}

\begin{description}
	\item [Vorteile:] \hfill
	\begin{enumerate}
		\item Enthält alles was die Kostenlose Variante \ref{subsubsec:Free} kann. 
		\item 3,000 Minuten im Monat für die Nutzung von Quellcode-Test-Tools für private Repositories.
		\item 2 GB Speicher für GitHub-Pakete.
		\item Informelle Pull-Anfragen starten, bevor man die formelle Überprüfung durchführt.
		\item Einschränkungen für das Zusammenführen von Branchen erzwingen.
		\item Anforderung zur Review durch ausgewählte Benutzer.
		\item Rechte für die Branchen geben, so können nur bestimmte Benutzer an einer bestimmten Branche arbeiten.
	\end{enumerate}
	\item [Nachteile:] \hfill
	\begin{enumerate}
		\item Monatlichen Beitrag in Höhe von 4\$ pro Benutzer.
		\item Hosten auf eigenem Server ist in diesem Paket nicht erhältlich.
	\end{enumerate}
\end{description}

Außerdem hat Github eine Desktop-Anwendung, mit der verschiedene Operationen wie (Diffs anzeigen, fetchen, mergen, pushen) ohne Command-line durchgeführt werden können.

