\section{Grundlagen}
\label{sec:Grundlagen}

\subsection{Definition}
\label{subsec:Definition}

\begin{description}
	\item [Codereview:]
		Unter Codereview Versteht man, eine manuelle Überprüfung des Quellcodes durch andere Entwickler als den Autor. Es gilt als wertvolles Werkzeug zur Reduzierung von 							Softwarefehlern und zur Verbesserung der Qualität von Softwareprojekten \cite{bacchelli2013expectations}.

	\item [VCS:]
		"`Versionskontrollsysteme sind Softwaretools, mit deren Hilfe Softwareteams Quellcodeänderungen verwalten können. Die Versionskontrollsoftware verfolgt jede Änderung am Code und 		speichert sie in einer speziell hierfür angelegten Datenbank. Unterläuft einem Entwickler ein Fehler, kann er einen Schritt zurück machen, seinen Code mit früheren Codeversionen 		abgleichen und Korrekturen implementieren. Und das mit minimalen Beeinträchtigungen für seine Teamkollegen."' \cite{version-control-System}

	\item [Git:]
		Git ist eine freie und Open-Source Software zur Versionsverwaltung von Dateien, das 2005 ursprünglich von Linus Torvalds, dem berühmten Entwickler des Linux Betriebssystem-				Kernel, entwickelt wurde.
	
	\item [SVN:]
		Apache Subversion ist eine freie Software zur zentralen Versionsverwaltung von Dateien und Verzeichnissen. Die Versionierung erfolgt in einem zentralen Repository in Form einer  		Revisionszählung.
	
	\item [CVN:]
		\unsure{Keine eindeutige Definition. Frag Björn.}
	
	\item [Post-commit:]
		Die Überprüfung erfolgt, nachdem die Änderungen am Quelltext an das Ziel-Repository gesendet wurde.
		\begin{description}
			\item [Vorteile:] \hfill
			\begin{enumerate}
				\item Andere Teammitglieder sehen die Codeänderungen und können ihre Arbeit entsprechend ändern
				\item Dieser Prozess kann praktisch sein, weil einige Änderungen komplex und lang sein können und mehrere Schritte für den Review erfordern
			\end{enumerate}
			
			\item[Nachteile:] \hfill
			\begin{enumerate}
				\item Erhöhte Wahrscheinlichkeit, dass schlechter Code in das Haupt Repository gelangt, was sich auf die Arbeit des gesamten Teams auswirkt
				\item Wenn Fehler gefunden werden, kann es eine Weile dauern, bis der Entwickler zu dem Modul zurückkehrt, an dem er gearbeitet hat
			\end{enumerate}
		\end{description}
		
		
	\item [pre-commit:]
		Ist eine Art des Reviews, mit der man die Änderungen des Quelltextes überprüft, bevor die in das Haupt-Repository des Versionsverwaltungssystem gelangen.
		\begin{description}
			\item [Vorteile:] \hfill
			\begin{enumerate}
				\item Diese Art erstellt eine Art von Sicherheit, da fehlerhafte oder unvollständige Änderungen durch den Review auftreten können
				\item Dieser Prozess stellt sicher, dass der Autor seine Änderungen überprüfen lässt und nicht direkt in das Ziel-Repository eincheckt
			\end{enumerate}
		\end{description}
		
	\item [Pull-Request:]
		Anfrage erstellen und an die Reviewer abschicken um Änderungen am Code zu überprüfen. Meistens arbeitet der Entwickler an einem Zweig vom master-branch und sobald er fertig ist 			sendet er die Pull-Request an die Review. Wird eine Pull-Request akzeptiert, so spricht man von einem Merge, wird er geschlossen, so spricht man von einem Close.
	
	\item [Continuous Integration(CI):]
		"`CI bedeutet Continuous Integration, also der Automatisierungsprozess für Entwickler. Bei einer erfolgreichen CI werden regelmäßig neue Codeänderungen für Apps entwickelt, 				geprüft und in einem gemeinsamen Repository zusammengeführt. Damit soll der Konflikt verhindert werden, den zu viele Branches einer App verursachen können, wenn sie zeitgleich 			entwickelt werden."' \cite{RedHat}
			
	\item [Continuous Delivery (CD):]
		"`Continuous Delivery bedeutet üblicherweise, dass App Änderungen eines Entwicklers automatisch auf Bugs getestet und in ein Repository (wie GitHub oder eine Container 					Registry) hochgeladen werden, von wo aus sie vom Operations-Team in einer Live-Produktivumgebung bereitgestellt werden können. Dieser Vorgang ist die Antwort auf Transparenz- 				und Kommunikationsprobleme zwischen Dev- und Business-Teams. Damit soll sichergestellt werden, dass neuer Code mit minimalem Aufwand implementiert werden kann."' \cite{RedHat}
		
	\item [Continuous Deployment(CD):]
		"`Continuous Deployment (das andere „CD“) kann sich auf die automatische Freigabe von Entwickleränderungen vom Repository zur Produktivphase beziehen, wo sie direkt vom Kunden 			genutzt werden können. Dieser Vorgang soll der Überlastung von Operations-Teams bei manuellen Prozessen entgegenwirken, die die Anwendungsbereitstellung verlangsamen. Continuous 		Development baut die Vorteile der Continuous Delivery aus, indem auch noch die nächste Phase der Pipeline automatisiert wird."' \cite{RedHat}

	Dieser Prozess kann als eine Pipeline vorgestellt werden, die die \cref{fig:RedHat} beschreibt.
	\begin{figure}[H]
		\centering
		\includegraphics[width=1.0\textwidth]{Red Hat CI CD}
		\caption[\ac{CI}/\ac{CD}]{\ac{CI}/\ac{CD} Prozess als Pipeline\\ \cite{RedHat}}
		\label{fig:RedHat}
	\end{figure}
	
\end{description}

\subsection{Motivation}
\label{subsec:Gründe}
Durch den Review des Codes bestehen zahlreiche Vorteile. Davon sind:

\begin{enumerate}
	\item Das Vermeiden von Fehlern im Quellcode, denn es ist immer sehr umständlich, wenn erst nach einer Weile einen Fehler auftritt, was eine ältere Änderung verursachte.
		Fehler müssen nicht nur, diejenige die das Programm nicht mehr zum Laufen bringen. Die können auch Falsche Schnittstellenspezifikation oder Verletzung von Namenskonventione 
		etc.\ sein.
	\item Fehler die erst beim Kunden auftreten, weil vorher keiner drüber geguckt hat kosten Geld und Reputation.
	\item Die Reviewer können durch lesen der Änderungen lernen.
	\item Der Autor kann durch Review-Kommentare lernen.
	\item Verbesserungsvorschläge und vielmehr.
\end{enumerate}

\subsection{Vorgehensmethoden}
\label{subsec:Vorgehensmethoden}
Es gibt verschiedene Verfahren von Reviews. Die am häufigsten benutzte Verfahren sind:

\begin{itemize}
	\item Over–the–shoulder: Der Autor setzt direkt neben dem Reviewer, der die Änderungen kritisiert. Dieses Verfahren ist eine informelle Review.
	\item Email pass-around: Der Autor sendet eine E-Mail mit dem Code zur Überprüfung des Codes an die Reviewer.
	\item Tool-assisted: auch \ac{CRS}. Hier werden Tools benutzt, die die Änderungen je nach Einstellung des Systems an die Rezensenten zum Überprüfen weiterleiten.
	\item Reviewsitzung: Das Team trefft sich regelmäßig, je nach Absprache. Diverse Meinungen werden dargestellt.
	\item Stand Up Meeting: Am Ende des Tages werden die Änderungen diskutiert. Das Team muss nicht unbedingt dabei sein, sondern nur die für die Änderungen zuständig sind.
	\item Planung: Entwickler werden zur Review eingeladen, rollen werden verteilt und die Bedingungen werden festgelegt.
\end{itemize}

\textbf{Nachteile der Methoden die nicht Tool-basiert sind:}
\begin{itemize}
	\item Die Änderungen werden zuerst im Quellcode geschrieben, was Fehler enthalten kann und danach kommt die Review.
	\item Dieser Prozess ist nicht flexibel, denn es fordert die Anwesenheit der für die Änderungen zuständige Personen zusammen mit den Rezensenten.
	\item Es wird für den Review keine Dokumentation erstellt.
\end{itemize}

\subsection{Mögliche Abläufe des Reviews}
\label{sec:review abläufe}

\unsure{beschreibe die Workflows die du verwendest}
\unsure{du kannst auch von here z. B. auf das Bild verweisen}