\section{Grundlagen}

\subsection{Definition}

\textbf{Was versteht man unter Codereview} ?

Darunter versteht man ein Verfahren, das die Gelegenheit jemandem anderen anbietet, die Änderungen des Autors im Quelltext zu überprüfen.
Es ist eine Art der Sicherheit, um Fehler zu vermeiden, jedoch nicht nur.

\subsection{Warum überhaupt}
Durch die Review des Codes bestehen zahlreiche Vorteile. Davon sind:

\begin{enumerate}
\item Das Vermeiden von Fehler im Quellcode, denn es ist immer sehr umständlich, wenn erst nach einer Weile einen Fehler auftritt, was eine ältere Änderung verursachte.
	  Fehler müssen nicht nur, diejenige die das Programm nicht mehr zum Laufen bringen. Die können auch Falsche Schnittstellenspezifikation oder Verletzung von Namenskonventione 				  \ac{etc.} sein.
\item Fehler die erst beim Kunden auftreten, weil vorher keiner drüber geguckt hat kosten Geld und Reputation.
\item Die Reviewer können durch lesen der Änderungen lernen.
\item Der Autor kann durch Review-Kommentare lernen.
\item Verbesserungsvorschläge und vielmehr !
\end{enumerate}

\subsection{Vorgehensmethoden}
Es gibt verschiedene Verfahren von Reviews. Die am häufigsten benutzte Verfahren sind:

\begin{itemize}
\item Over–the–shoulder: Der Autor setzt direkt neben dem Reviewer, der die Änderungen kritisiert. Dieses Verfahren ist eine informelle Review.
\item Email pass-around: Der Autor sendet eine E-Mail mit dem Code zur Überprüfung des Codes an die Reviewer.
\item Tool-assisted: auch \ac{CRS}. Hier werden Tools benutzt, die die Änderungen je nach Einstellung des Systems an die Rezensenten zum Überprüfen weiterleiten.
\end{itemize}
