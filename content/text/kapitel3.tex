\section{Der Einfluss vom Codereview auf die Softwarequalität in den letzten Jahren}
\label{sec:Einfluss des Reviews}
Smartbear \cite{smartbear} führte im Jahr 2019 mit 1100 Softwareentwickler/Tester, IT- / Betriebsfachleute und Führungskräfte in 35 verschiedenen Branchen eine Umfrage durch. Die Teilnehmer der Umfrage arbeiten in Unternehmen mit unterschiedlichen Größen, von weniger als 25 Mitarbeitern bis zu über 10.000. Ähnliche Umfragen hat Smartbear regelmäßig durchgeführt.

Seit 2016 haben die Befragten identifiziert, dass das Codereview die beste Methode zur Verbesserung der Codequalität ist. Das Unit Testing im Jahr 2019 wurde als zweithöchste Wahl betrachtet und die Daten in der selben Rangliste vom Jahr 2019 zeigen: 25\% der Befragten wählen erneut das Codereview, was in der \cref{fig:Codereview-Softwarequalität} zu sehen ist. Die Wahrnehmung des Codereviews als Schlüssel für eine bessere Softwarequalität wurde durch die Daten vom Jahr 2019 bestätigt.

\begin{figure}[H]
	\centering
	\includegraphics[width=1.0\textwidth]{Codereview-Softwarequalität}
	\caption[Einfluss des Codereviews auf die Softwarequalität]{Codereview-Softwarequalität\\ \cite{smartbear}}
	\label{fig:Codereview-Softwarequalität}
\end{figure}

\section{Einfluss der investierte Zeit im Review}
\label{sec:reviewZeit}

Wie im \cref{sec:Einfluss des Reviews} erwähnt, dass Codereview und Softwarequalität eng verbunden sind, so dass es sich lohnt, die Zeit für das Review zu nehmen, um die Qualität des Quelltexts zu verbessern. Die Umfrage von Smartbear zeigt welche sind die wichtigsten Vorteile, die das Review mit sich bringt. Zwar ist das Auffinden von Fehlern die Hauptmotivation, jedoch kann dieser Prozess dabei helfen, die Kommunikation des Teams zu verbessern, das Wissen zu übertragen und die Standardisierung zu erhöhen \cite{smartbear}. 

Die Folgende \cref{fig:Vorteile des Codereviews} stellt die Wichtigste Vorteile des Reviews dar.

\begin{figure}[H]
	\centering
	\includegraphics[width=1.0\textwidth]{Vorteile des Codereviews}
	\caption[Vorteile des Codereviews]{Die wichtigsten Vorteile des Codereviews\\ \cite{smartbear}}
	\label{fig:Vorteile des Codereviews}
\end{figure}

Die verbesserte Softwarequalität und die Wissensübertragung im gesamten Team sind seit 2015 die beiden wichtigsten Vorteile \cite{smartbear}.

\subsection{Die vom Programmierer geplante Zeit für das Review}
\label{subsec:reviewerZeit}

Laut der Umfrage von Smartbear (siehe \cref{sec:Einfluss des Reviews}) \textbf{67\%} der Befragten nehmen mindestens wöchentlich am Review teil, während \textbf{42\%} überprüfen den Code täglich.

\subsection{Die am häufigsten für das Review verwendeten Methoden}
\label{subsec:Die am häufigsten verwendete Methoden}

Diese drei Methoden Ad-hoc, Meeting-basiert und Tool-basiert sind die gängigsten Ansätze zum Codereview.
Die entsprechende Daten von der Smartbears Umfrage \cite{smartbear} dieser Methoden verdeutlicht das Balkendiagramm \cref{fig:ReviewZeit} 

\begin{figure}[H]
	\centering
	\includegraphics[width=0.7\textwidth]{ReviewZeit}
	\caption[Die Verwendung der bekannten Reviews Methoden]{Die Verwendung der bekannten Reviews Methoden\\\cite{smartbear}}
	\label{fig:ReviewZeit}
\end{figure}

\begin{enumerate}
	\item \textbf{49\%} der Befragten führen mindestens wöchentlich das Review durch Ad-hoc oder Over-the-Shoulder durch, davon sind \textbf{20\%}, die ihr Code auf diese Weise 					täglich überprüfen.
	\item \textbf{44\%} der Befragten basiert ihr wöchentliches Review auf Tools, davon sind \textbf{29\%}, die das Review mit Tools täglich machen.
	\item \textbf{16\%} der Befragten organisieren ein Meeting wöchentlich für das Review, davon sind \textbf{3\%}, die das Review auf diese Weise täglich durchführen.
\end{enumerate}

\pagebreak

\section{Codereview-Tools}
\label{sec:Coderview-Tools}

Es gibt 4 Bereiche, mit denen Entwicklern ihren Code überprüfen.

\begin{itemize}
	\item \textbf{Repository-Verwaltungstools:} Diese Tools sind für die Verwaltung von Repositories zuständig. Entwickler können das Review durchführen, indem sie das Pull Request an.
		Teammitglieder senden oder auch durch andere Verfahren wie der Fall bei Helix Teamhub im \cref{subsec:HelixTeamHub}
		Beispiele dafür sind z.B.:
		\begin{itemize}
			\item Bitbucket \cref{subsec:Bitbucket}.
			\item Github \cref{subsec:Github}.
		\end{itemize}
	
		Smartbears Umfrage zufolge verwenden \textbf{69\%} der Befragten Repository Tools als Teil ihres Codereviews, unabhängig davon, ob sie das Review durch eine Pull Request 					durchführen oder in ein spezielles Peer Codereview Tool oder statisches Analysetool integrieren. Das bildet ein deutlicher Anstieg von \textbf{42\%} im Jahr 2018 							\cite{smartbear}

	\item \textbf{\ac{IDE}:} Wie z.B. Visual Studio\footnote{\url{https://visualstudio.microsoft.com}} bietet die Möglichkeit, direkt in der Umgebung das Review zu starten.

	\item \textbf{Statische Analysetools:} Ein Beispiel dafür ist SonarQube\footnote{\url{https://www.sonarqube.org}}. Es ist eine Plattform, die automatisierte Regeln für die statische Code-Analyse definiert. Der Autor wird benachrichtigt, dass jemand sein Pull Request bzw. seine Änderungen überprüft hat, das durch Repository-Verwaltungstools erstellt wurden.

	\item \textbf{Peer-Codereview-Tools:} Beispiele sind: Crucible im \cref{subsec:Crucible} und Gerrit im \cref{subsec:Gerrit}. Das Hauptziel solcher Tools ist, einen einfachen und 				unkomplizierten Review anzubieten. Sie können kein Repository verwalten, haben aber andere Eingenschaften beispielsweise Änderungsvergleich anzeigen.
\end{itemize}

\subsection{Codereview-System}
\label{subsec:CRS}

Ein Codereview-System ist ein allgemeiner Begriff, der ein Tool beschreibt, das fähig ist die Entwickler die Möglichkeit anbietet ein Codereview durchführen zu können und dieses Review zu steuern. Dieses System kann ein Repository Verwaltungstool, Statisches Analysetool, IDE oder ein Peer-Codereview-Tool sein. Beispiele dafür sind im \cref{sec:CRS-Git}.

\subsection{Kriterien \& Erwartungen an das System}
\label{sec:kriterien}

Folgende Eingenschaften sind die Kriterien, die in Betracht bei der Vorstellung von allen \acp{CRS} genommen werden sollen:

\begin{itemize}
	\item Es soll das \ac{VCS} Git unbedingt unterstützen.
	\item Die Unterstützung vom \ac{VCS} \ac{SVN} ist ein Bonus.
	\item Es soll den Änderungsvergleich sowohl inline als auch Side-by-side deutlich darstellen können.
	\item \ac{CI}/\ac{CD} Tools sollen in den Workflow eingebaut werden können oder in dem System integrieren sein.
	\item Das System kann auf dem eigenen Server gehostet werden.
	\item Das System kann die Review-Kommentare nach Beenden des Reviews automatisch mit dem Haupt-Repository Zusammenführen.
	\item Review-Kommentare sind auf Zeilenebene möglich.
	\item Zugriffsverwaltung ist ein Vorteil.
\end{itemize}
