\section{Fazit}

Dass das Codereview die bekannteste Methode ist, um die Qualität des Codes zu verbessern, hat sich erwiesen. Sie wurde am Anfang als Methode für das Auffinden von Fehlern am Quelltext vorgesehen, jedoch hat sie bei der Verbesserung der Codequalität beigetragen. Die moderne Codereview-Tools wie Gerrit, die über qualitative und hochwertige Daten verfügen, haben durch ihre Leichtigkeit die strengen Kriterien von den formellen Codereview-Methoden erleichtert \cite{mcintosh2016empirical}.

Zu guter Letzt sind die folgende Aussagen 

\begin{itemize}
	\item Jeder Entwickler soll seine Änderungen am Quelltext überprüfen lassen
	\item Die für das Review genommene Zeit ist Wert
	\item Das Review hilft bei der Verbesserung der Codequalität
	\item Das moderne Review durch nutzen ein \ac{CRS} ist sinnvoller als die formellen Methoden wie z.B. Ad-hoc \cref{subsec:Vorgehensmethoden}
\end{itemize}

von dem Experte Björn Schäpers \cite{Bjoern} anhand seiner Erfahrungen bestätigt.

Der Prozess der Überprüfung ist von einem Tool zum anderen unterschiedlich. In unserem Fall haben wir uns für das \ac{CRS} Gerrit \cref{subsec:Gerrit} aus den Grunden im \cref{subsec:Auswertung} entschieden, das der Prozess des Reviews in einer anderen und einfachen Form als die anderen Tools anbietet.

\label{seitenreinschrifft}
